\documentclass{article}

\usepackage{amsmath, amssymb}
\usepackage{xcolor}
\usepackage{caption, subcaption}
\usepackage{graphicx}
\usepackage{bm}             % For bold math symbols
\usepackage{soul}
\usepackage{outlines}   % For multilevel lists using the outline environment
\usepackage{tikz}
\usetikzlibrary{arrows.meta, positioning}


\usepackage[semicolon]{natbib}

\usepackage{hyperref}   % For the \url{} command


\newcommand{\bG}{\mathbb{G}}
\newcommand{\bF}{\mathbb{F}}
\newcommand{\bP}{\mathbb{P}}
\newcommand{\bE}{\mathbb{E}}
\newcommand{\bV}{\mathbb{V}}

\newcommand{\zdens}{\frac{e^{-z^2/2}}{\sqrt{2\pi}}}
\newcommand{\indep}{\perp \!\!\! \perp}

\newcommand{\logit}{\mathrm{logit}}


\title{Analytical Uncertainty Quantification for Multilevel Mediation Analysis}

\begin{document}

\maketitle
\section{Notes to Self}
\begin{outline}
    \1 Summarize problem
    \1 Describe MSEP formula and decomposition
    \1 Give conditional variance formula 
        \2 Explain derivation
        \2 Explain components of formula
    \1 Justify neglecting correction term
\end{outline}

This note summarizes the work of \citet{Boo98} as it pertains to Rowin's thesis.

Our statistical model is a GLM with random effects. We assume that our response, $Y$, has an exponential family distribution with the canonical parameterization. That is,
%
\begin{equation}
    f(y) = \exp\left( \left[y \theta - b(\theta)\right] + c(y) \right) \label{eq:exp_dens}
\end{equation}
%
If $Y$ is Bernoulli$(p)$, then $\theta = \log[p/(1-p)]$, $b(\theta) = \log(1 + e^\theta) = -\log(1-p)$, and $c(y) = 0$. We model the parameter $\theta$ as depending on two sets of covariates, $X$ and $Z$, through some fixed effects, $\beta$, and random effects, $U$, respectively. We assume that the observed responses occur in groups, or clusters. Write $Y_{ij}$ for the $j$th observation from group $i$. We further assume that each group has its own level for the random effect, $U_i$, and that these random effects are independent and identically distributed from some common distribution, usually Normal$(0, \Sigma)$.


Since we're doing regression, we will focus on the conditional mean of $Y$ given covariates. Since we're also doing mixed-effects modelling, we start by further conditioning on the random effects, $U$. To this end, write $\mu_{ij} = \bE (Y |U = U_i, X = X_{ij}, Z = Z_{ij})$ for the mean of observation $j$ from group $i$, when the random effects and covariates are held fixed. Following the usual GLM setup, we model $g(\mu_{ij}) = \eta_{ij} = X_{ij} \beta + Z_{ij} U_i$. 

As an aside, I've always had trouble keeping all the different scales which arise in GLMs straight in my head. There's $\mu$, the mean of $Y$, $\eta$, the linear predictor, and $\theta$, the canonical parameter\footnote{Note that this canonical parameter often differs from the parameters we're used to thinking about for the distribution in question. See, e.g., the Bernoulli distribution above, or try expressing the Normal$(\mu, \sigma)$ as an exponential family distribution.}. There are then functions which map between these different scales: the link function, $g(\mu) = \eta$, the mean function, $b'(\theta) = \mu$, and the ``variance function'', $b''(\theta)$. This last one bothers me because if you include nuisance (or overdispersion) parameters in your exponential family then the variance of $Y$ is not equal to the variance function. If you choose the canonical link function though, then $g(\mu) = \theta$ (this is the definition of the canonical link), which really muddies things for me. Rant over.

Our current goal is the estimate/predict the values of the random effects within each group. It is common in the literature to use the word ``predict'' here, because the random effects aren't parameters to be estimated, but instead random variables, which we would usually talk about ``predicting''. Note that the different groups in our problem are independent, so we can focus on prediction of a single $U_i$, then apply our method individually to the other groups. \citeauthor{Boo98} recommend predicting $U_i$ with its conditional mean given the data from group $i$, $Y_i$. Note that this conditional mean depends on the model parameters, $\beta$ and $\Sigma$, which in practice must be estimated. A lot of software (e.g. the \texttt{lme4} package in \texttt{R}) actually uses the conditional mode instead of the conditional mean, since there is a nice algorithm to compute the mode, but the mean is hard (i.e. slow) in general.

\citeauthor{Boo98} discuss prediction of the linear predictor, $\eta_{ij}$, but we can just think of this as $U_i$.

\section{Interpretation of Equation (15)}

There are a bunch of symbols in Equation 15 of \citeauthor{Boo98} which we define here. First, $Z_i$ is the observed random effects covariance matrix in group $i$. Next, $z_i$ is the contrast for the random effects in group $i$ at which we want predictions. E.g. If we want to predict $u_i[1]$ then $z_i = (1, 0, \ldots, 0)$. Next, $G$ is the covariance matrix of $u_i$. That is, the covariance matrix of the random effects within a group. Note that $G$ must be estimated. Finally, $\bar{W}_i$ is a bit complicated. Formally, it is the matrix of weights for the iteratively re-weighted least squares algorithm, evaluated at the predicted random effects in group $i$, $\hat{u}_i$. Let's unpack that.

In the bottom paragraph of the first column on page 265, \citeauthor{Boo98} define the ``iterative weights'' as $w_{ij}/ \left[ \sigma_0^2 V(\mu_{ij}) g'(\mu_{ij})^2 \right]$. For our purposes, both $w_{ij}$ and $\sigma_0^2$ can be ignored (i.e. set to 1). Now onto $V(\mu)$ and $g'(\mu)$. The former, called the ``variance function'', can equivalently be expressed as $b''(\theta)$, where $b$ comes directly from the exponential family density given in \eqref{eq:exp_dens}. Differentiating $b$ twice, then returning to $\mu$-scale, we get that $V(\mu) = \mu (1 - \mu)$. Next, $g$ is the link function, which maps $\mu$ to the linear predictor scale. For logistic regression, $g(\mu) = \logit(\mu)$. Differentiating, we get $g'(\mu) = [\mu (1 - \mu)]^{-1}$. Putting this all together, we get that the ``iterative weight'' for a single observation in the logistic model is $\mu_{ij} (1 - \mu_{ij})$.

Returning now to Equation (15) of \citeauthor{Boo98}, we get that $\bar{W}_i$ is a diagonal matrix with $jj$th element equal to $\hat{\mu}_{ij} (1 - \hat{\mu}_{ij})$. Note that the hat here denotes both estimation of parameters and prediction of $u_{ij}$ by its conditional mean given $y_i$.






\bibliographystyle{plainnat}
\bibliography{Rowin_Bib}

\end{document}